% units.tex
% Edward F Brown, Michigan State University
% 
% typesetting of dimensional values
% Features: proper spacing between value and unit; unit is set in correct type

% powers
\newcommand*{\ee}[1]{\ensuremath{\times 10^{#1}}}		% 3.0\ee{4} => 3.0\times 10^{4}
\newcommand*{\sci}[2]{\ensuremath{#1\ee{#2}}}			% \sci{3.0}{4} => 3.0\times 10^{4}
\newcommand{\power}[2]{\ensuremath{{#1}^{#2}}}		    % \power{2}{3} => 2^{3}

% per APS style, there is a full space between value and unit, and a thin space between unit components
\newcommand*{\valueskip}{\;}
\newcommand*{\vsk}{\valueskip}
\newcommand*{\unitskip}{\,}
\newcommand*{\usk}{\unitskip}
\newcommand*{\val}[2]{\ensuremath{#1\valueskip#2}}

% sometimes you want to input a range of numbers
% example: \rng{2}{3} puts 2 to 3; \rng[--]{2}{3} puts 2--3, where the two dashes are correctly typeset 
% as an en-dash rather than as a minus sign
\newcommand*{\rng}[3][~to~]{\ensuremath{#2\textrm{#1}#3}}
% this puts in a range with a unit
% example: \valrng[--]{2}{3}{\times \val{10^{3}}{\kilo\gram}}
\newcommand*{\valrng}[4][~to~]{\ensuremath{\left(#2\textrm{#1}#3\right)#4}}


% units should be in upright (roman) font
\newcommand*{\unitstyle}[1]{\mathrm{#1}}

% pre-defined macros for common units
% prefixes
\newcommand*{\yocto}{\unitstyle{y}}		% 10^{-24}
\newcommand*{\zepto}{\unitstyle{z}}		% 10^{-21}
\newcommand*{\atto}{\unitstyle{a}}		% 10^{-18}
\newcommand*{\femto}{\unitstyle{f}}		% 10^{-15}
\newcommand*{\pico}{\unitstyle{p}}		% 10^{-12}
\newcommand*{\nano}{\unitstyle{n}}		% 10^{-9}
\newcommand*{\micro}{\unitstyle{\mu}}	% 10^{-6}
\newcommand*{\milli}{\unitstyle{m}}		% 10^{-3}
\newcommand*{\centi}{\unitstyle{c}}		% 10^{-2}
\newcommand*{\deci}{\unitstyle{d}}		% 10^{-1}
%
\newcommand*{\deka}{\unitstyle{da}}		% 10^{1}
\newcommand*{\hecto}{\unitstyle{h}}		% 10^{2}
\newcommand*{\kilo}{\unitstyle{k}}		% 10^{3}
\newcommand*{\Mega}{\unitstyle{M}}		% 10^{6}
\newcommand*{\Giga}{\unitstyle{G}}		% 10^{9}
\newcommand*{\Tera}{\unitstyle{T}}		% 10^{12}
\newcommand*{\Peta}{\unitstyle{P}}		% 10^{15}
\newcommand*{\Exa}{\unitstyle{E}}		% 10^{18}
\newcommand*{\Zetta}{\unitstyle{Z}}		% 10^{21}
\newcommand*{\Yotta}{\unitstyle{Y}}		% 10^{24}

% base units, mks
\newcommand*{\meter}{\unitstyle{m}}     % SI length
\newcommand*{\kilogram}{\kilo\gram}     % SI mass
\newcommand*{\second}{\unitstyle{s}}    % SI time

\newcommand*{\Kelvin}{\unitstyle{K}}
\newcommand*{\K}{\Kelvin}  %degrees Kelvin

% base units, cgs
\newcommand*{\cm}{\centi\meter}         % cgs length
\newcommand*{\gram}{\unitstyle{g}}      % cgs mass

% derived units
\newcommand*{\grampercc}{\gram\unitskip\power{\cm}{-3}} % mass density
\newcommand*{\grampersquarecm}{\gram\unitskip\power{\cm}{-2}} % column depth
\newcommand*{\GramPerCc}{\grampercc}
\newcommand*{\GramPerSc}{\grampersquarecm}
\newcommand*{\columnunit}{\grampersquarecm}     % mass column density
\newcommand*{\dyne}{\unitstyle{dyn}}            % dyne
\newcommand*{\erg}{\unitstyle{erg}}             % ergs
\newcommand*{\ergs}{\erg}
\newcommand*{\gauss}{\unitstyle{G}}             % gauss
\newcommand*{\ergspersecond}{\erg\unitskip\power{\second}{-1}}  % luminosity
\newcommand*{\ergspergram}{\erg\unitskip\power{\gram}{-1}}  % energy per mass
\newcommand*{\cgsflux}{\erg\unitskip\power{\cm}{-2}\unitskip\power{\second}{-1}} % flux

% Nuclear and atomic units
\newcommand*{\amu}{\unitstyle{u}}               %atomic mass unit
\newcommand*{\angstrom}{\mbox{\AA}}             %Angstrom
\newcommand*{\fermi}{\femto\meter}              %fermi
\newcommand*{\eV}{\unitstyle{eV}}               %eV
\newcommand*{\keV}{\kilo\eV}                    %keV
\newcommand*{\MeV}{\Mega\eV}                    %MeV
\newcommand*{\GeV}{\Giga\eV}                    %GeV
% Beam units
\newcommand*{\MeVA}{\MeV/A}                     % MeV per nucleon
\newcommand*{\GeVA}{\GeV/A}                     % GeV per nucleon

% solar and astronomical units
\newcommand*{\Msun}{\ensuremath{M_\odot}}       % solar mass
\newcommand*{\Lsun}{\ensuremath{L_\odot}}       % solar luminosity
\newcommand*{\Rsun}{\ensuremath{R_\odot}}       % solar radius
\newcommand*{\yr}{\unitstyle{yr}}               % Calendar year, 365 d
\newcommand*{\julyr}{\unitstyle{a}}             % Julian year = 365.25 * 86400 s
\newcommand*{\Myr}{\Mega\yr}                    % $10^{6}\,\yr$
\newcommand*{\Gyr}{\Giga\yr}                    % $10^{9}\,\yr$
\newcommand*{\AU}{\unitstyle{au}}               % astronomical unit
\newcommand*{\parsec}{\unitstyle{pc}}           % parsec
\newcommand*{\kpc}{\kilo\parsec}                % kiloparsec
\newcommand*{\Jansky}{\unitstyle{Jy}}           % Jansky
\newcommand*{\mJy}{\unitstyle{\mu Jy}}          % micro Jansky
\newcommand*{\Msunperyr}{\Msun\unitskip\power{\yr}{-1}}	% solar masses per year

% misc. units
\newcommand*{\minute}{\unitstyle{m}}            % minute = 60 s
\newcommand*{\hour}{\unitstyle{h}}              % hour = 60 * 60 s
\newcommand*{\aday}{\unitstyle{d}}              % IAU definition: d = 86400 s (\day aleady in use)
\newcommand*{\km}{\kilo\meter}                  % kilometers
\newcommand*{\Hz}{\unitstyle{Hz}}               % Hertz
\newcommand*{\ksec}{\kilo\second}               % kilosecond
\newcommand*{\mol}{\unitstyle{mol}}             % mole
\newcommand*{\barn}{\ensuremath{\mathrm{b}}}    % barn
