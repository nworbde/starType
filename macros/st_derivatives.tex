% derivatives.tex
% Edward F Brown, Michigan State University
% 
% typesetting of common derivatives
\ifthenelse{\boolean{@italic_dif}}
{
    \typeout{setting italic derivative symbol}
    \newcommand*{\dif}{\ensuremath{d}}  % differential operator, italic typeface
    \newcommand*{\Dif}{\ensuremath{D}}
}{
    \typeout{setting roman derivative symbol}
    \newcommand*{\dif}{\ensuremath{\mathrm{d}}}  % differential operator, roman typeface
    \newcommand*{\Dif}{\ensuremath{\mathrm{D}}}
}
%derivatives
\newcommand*{\DD}[2]{\mathchoice{\frac{\dif #1}{\dif #2}}{\dif #1/\dif #2}{\dif #1/\dif #2}{\dif #1/\dif #2}} % total derivative
\newcommand*{\dd}[2]{\mathchoice{\frac{\partial #1}{\partial #2}}{\partial#1/\partial#2}{\partial#1/\partial#2}{\partial#1/\partial#2}} % partial derivative
\newcommand*{\jac}[4]{\mathchoice{\frac{\partial(#1,#2)}{\partial(#3,#4)}}{\partial(#1,#2)/\partial(#3,#4)}{\partial(#1,#2)/\partial(#3,#4)}{\partial(#1,#2)/\partial(#3,#4)}} %jacobian
\newcommand*{\tderiv}[3]{\mathchoice{\left(\frac{\partial #1}{\partial #2}\right)_{#3}}{\left(\partial #1/\partial #2\right)_{#3}}{\left(\partial #1/\partial #2\right)_{#3}}{\square}} %thermodynamic derivative

% common shortcuts
\newcommand*{\ddt}[1]{\dd{#1}{t}}    % partial time derivative 
\newcommand*{\DDt}[1]{\DD{#1}{t}}    % total time derivative
\newcommand*{\ddx}[1]{\dd{#1}{x}}    % partial derivative wrt x
\newcommand*{\DDx}[1]{\DD{#1}{x}}    % total derivative wrt x
\newcommand*{\ddy}[1]{\dd{#1}{y}}    % partial derivative wrt y
\newcommand*{\DDy}[1]{\DD{#1}{y}}    % total derivative wrt y
\newcommand*{\ddz}[1]{\dd{#1}{z}}    % partial derivative wrt z
\newcommand*{\DDz}[1]{\DD{#1}{z}}    % total derivative wrt z
\newcommand*{\ddr}[1]{\dd{#1}{r}}    % partial derivative wrt r
\newcommand*{\DDr}[1]{\DD{#1}{r}}    % total derivative wrt r

\ifthenelse{\boolean{@st_vectors}}
{
    \typeout{using boldmath for vectors in fluid derivatives}
    \newcommand*{\derivVec}[1]{\bvec{#1}}
    \newcommand*{\derivGrad}{\grad}
    \newcommand*{\derivDot}{\vdot}
}{
    \typeout{option `vectors' is not selected; falling back on vanilla latex vec command}
    \newcommand*{\derivVec}[1]{\vec{#1}}
    \newcommand*{\derivGrad}{\nabla}
    \newcommand*{\derivDot}{\cdot}
}

\ifthenelse{\boolean{@pound_for_Lie}}                    % Lie derivative
{
    \typeout{using mathsterling for Lie derivative}
    \newcommand*{\LD}[1][]{\ensuremath{\mathsterling_{\derivVec{#1}}}}
}
{
    \typeout{using calligraphic L for Lie derivative}
    \newcommand*{\LD}[1][]{\ensuremath{\mathcal{L}_{\derivVec{#1}}}}
}

% convective derivative and fluid derivative
\newcommand*{\convectiveDerivative}[2][v]{\ddt{#2}+\derivVec{#1}\derivDot\derivGrad #2}
\ifthenelse{\boolean{@uppercase_convective}}
{
    \typeout{setting uppercase D for convective derivative}
    \newcommand*{\convectiveD}[1]{\mathchoice{\frac{\Dif #1}{\Dif t}}{\Dif#1/\Dif#2}{\Dif#1/\Dif#2}{\Dif#1/\Dif#2}}
}{
    \typeout{setting lowercase d for convective derivative}
    \newcommand*{\convectiveD}[1]{\DDt{#1}}
}
\newcommand*{\fluidDerivative}[2][v]{\ddt{#2} + \LD[#1]{#2}}
\newcommand*{\fluidD}[1]{\mathchoice{\frac{\Dif #1}{\Dif t}}{\Dif#1/\Dif t}{\Dif#1/\Dif t}{\Dif#1/\Dif t}}
