% derivatives.tex
% Edward F Brown, Michigan State University
% 
% typesetting of common derivatives
\ifthenelse{\boolean{@italic_dif}}
{
    \typeout{setting italic derivative symbol}
    \newcommand*{\dif}{\ensuremath{d}}  % differential operator, italic typeface
    \newcommand*{\Dif}{\ensuremath{D}}
}{
    \typeout{setting roman derivative symbol}
    \newcommand*{\dif}{\ensuremath{\mathrm{d}}}  % differential operator, roman typeface
    \newcommand*{\Dif}{\ensuremath{\mathrm{D}}}
}
\newcommand*{\jac}[4]{\ensuremath{\frac{\partial(#1,#2)}{\partial(#3,#4)}}} %jacobian
\newcommand*{\tderiv}[3]{\ensuremath{\left(\frac{\partial #1}{\partial #2}\right)_{#3}}} %thermodynamic derivative

%derivatives
\newcommand*{\ddt}[1]{\frac{\partial #1}{\partial t}}    % partial time derivative 
\newcommand*{\DDt}[1]{\frac{\dif #1}{\dif t}}            % total time derivative
\newcommand*{\ddx}[1]{\frac{\partial #1}{\partial x}}    % partial derivative wrt x 
\newcommand*{\DDx}[1]{\frac{\dif #1}{\dif x}}            % total derivative wrt x
\newcommand*{\ddy}[1]{\frac{\partial #1}{\partial y}}    % partial derivative wrt y 
\newcommand*{\DDy}[1]{\frac{\dif #1}{\dif y}}            % total derivative wrt y
\newcommand*{\ddz}[1]{\frac{\partial #1}{\partial z}}    % partial derivative wrt z 
\newcommand*{\DDz}[1]{\frac{\dif #1}{\dif z}}            % total derivative wrt z

\newcommand*{\dd}[2]{\frac{\partial #1}{\partial #2}}    % partial derivative

\newcommand*{\LD}[1][]{\mathcal{L}_{\bvec{#1}}}                              % Lie derivative
\newcommand*{\fluidD}[1]{\frac{\Dif #1}{\Dif t}}
